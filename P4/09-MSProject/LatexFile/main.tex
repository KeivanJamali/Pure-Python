\documentclass{titlepage}

%%%%%%%%%%%%%%% Defiding %%%%%%%%%%%%%%%%%%
\newcommand{\mytitle}{Physics-Informed Neural Networks for Spatio-Temporal Forecasting of Origin-Destination Demand in Urban Transportation Systems}
\newcommand{\myname}{کیوان جمالی}
\newcommand{\myfrofessor}{دکتر یوسف شفاهی}
\newcommand{\mydate}{فروردین 1404}
%%%%%%%%%%%%%%% Defiding %%%%%%%%%%%%%%%%%%
% Page style
\pagestyle{fancy}
\fancyhf{}  % Clear all headers and footers
\fancyhead[L]{\thepage}
\fancyhead[R]{\mytitle}
\renewcommand{\headrulewidth}{0.4pt}

%%%%%%%%%%%%%%%%%%%%%%%%%%%%%%%%%%%%%%%%%%%%%%%%%%%%%%%
% ------------------ Document ------------------
\begin{document}

\maketitle  % This will generate the custom title page

\pagenumbering{alph}
\setcounter{page}{1}

% ----------------- Abstract -------------------
\section*{چکیده}
این پژوهش با هدف توسعه‌ی یک چارچوب نوآورانه مبتنی بر داده برای پیش‌بینی تقاضای سفرهای شهری انجام می‌شود که در آن شبکه‌های عصبی عمین با اصول فیزیکی حاکم بر سیستم‌های حمل و نقل ترکیب شده‌اند. با بهره گیری از داده‌های  نظرسنجی در منطقه شهری شیراز برای سال‌های 1380، 1380 و 1380، این مطالعه به پیش‌بینی ماتیس ‌های مبدا-مقصد  در طول زمان می‌پردازد. روش پیشنهادی شامل طراحی یک شبکه عصبی آگاه از فیزیک  است که محدودیت‌های خاص دامنه، مانند تمرکز مکانی تقاضا درنزدیکی مناطق تجاری مرکزی و حفظ تعداد سفرها، را در فرایند یادگیری لحاظ می‌کند. با گنجاندن این مفروضات واقع‌گرایانه در ساختار شبکه عصبی، مدل به دقت پیش‌بینی بالا و همچنین تفسیرپذیری مناسب دست می‌یابد. این روش با استفاده از کتابخانه پایتورچ  پیاده سازی خواهد شد و با رویکردهای کلاسیک مانند مدل جاذبه‌ و روش‌های مبتنی بر آنتروپی مقایسه خواهد شد. نتایج مورد انتظار شامل مدلی مقاوم‌تر و تطبیق‌پذیر برای برنامه ریزی بلند مدت حمل و نقل است. نوآوری‌های کلیدی این پژوهش عبارت‌اند از: ادغام محدودیت‌های فیزیکی با یادگیری عمیق در مدل‌سازی حمل و نقل، استفاده از داده‌های نادر طولی در مورد سفرهای شهری، و امکان توسعه چارچوبی بنیادین برای بازاندیشی مدل سنتی چهار مرحله‌ای. 
\vfill
\newpage

% ----------------- Tables --------------------
% Now you can continue with your document content
\tableofcontents
\listoffigures
\listoftables
\clearpage

\pagenumbering{arabic}
\setcounter{page}{1}

% --------------- Introduction --------------------
\section{مقدمه و بیان مسئله}
در این بخش، ابتدا مقدمه‌ای کلی در مورد موضوع تحقیق آورده شده و سپس مسئله تحقیق به طور مشخص بیان می‌شود. در اینجا به ماشین لرنینگ \LTRfootnote{Machine Learning} میپردازیم

% ---------------- Litrature ---------------------
\section{مرور ادبیات فنی}
این مطالعه با هدف پیش‌بینی ماتریس‌های مبدا-مقصد از طریق ادغام قوانین فیزیکی رفتار سفر با چارچوب‌های یادگیری عمیق است، تا در نهایت دقت و تفسیرپذیری مدل‌های تقاضا بهبود یابد. برآورد دقیق تقاضای سفر نقش محوری در برنامه‌ریزی حمل و نقل، سرمایه‌گذاری زیرساختی و تدوین سیاست‌ها ایفا می‌کند. مدل‌های سنتی مانند مدل جاذبه یا روش فراتر  مبتنی بر فرضیات قوی و روابط ایستا هستند که ممکن است نتواند ماهیت پویا و غیرخطی رفتار سفر در طول زمان را به درستی بازتاب دهند. همچنین مدل‌های یادگیری ماشین مانند مدل‌های جنگل‌های تصادفی  و درختان تقویت شده با گرادیان   که با استفاده از فرایند خطی سازی پیش‌بینی می‌کنند و مدل‌های یادگیری عمیق مانند شبکه عصبی با اتصال کامل و شبکه عصبی بازگشتی که با استفاده از داده‌ها ارتباط پنهان بین داده‌ها و خروجی مربوطه را پیدا می‌کنند. هرچند این دسته از مدل‌ها در تفسیرپذیری دچار مشکل می‌باشند. با گسترش دسترسی به داده‌های کلان و پیشرفت روش‌های محاسباتی، نیاز به مدل‌هایی که بتوانند شواهد تجربی و دانش نظری را به صورت هم‌زمان در خود جای دهند، بیش از پیش احساس می‌شود. ادغام دانش تخصصی حوزه با شبکه‌های عصبی، فرصتی برای غلبه بر محدودیت‌های رویکردهای صرفا داده‌محور یا صرفا تحلیلی فراهم می‌آورد.
این پژوهش بر توسعه مدلی برای پیش‌بینی ماتریس‌های مبدا-مقصد با استفاده از داده‌های نظرسنجی در سه سال متمایز 1380، 1380 و 1380 تمرکز دارد. دامنه این مطالعه شامل موارد زیر است:
\begin{enumerate}
    \item سلام
\end{enumerate}
لازم به ذکر است که مدل‌سازی انتخاب نوع سفر، تخصیص مسیر و ارزیابی سیاست‌ها مستقیما در این مطالعه بررسی نخواهند شد، اما ممکن است در گسترش‌های آتی مورد توجه قرار گیرند.
پیش‌بینی دقیق جریان‌های مبدا=مقصد با چالش‌های متعددی روبه‌رو است، از جمله: غیرخطی بودن سیر تحول تقاضا، حساسیت به تغییرات کاربردی زمین و زیرساخت، و محدودیت‌های داده‌ای. مدل‌های موجود غالبا این پیچیدگی‌ها را ساده‌سازی می‌کنند و در نتیجه خروجی‌هایی کمتر دقیق یا گمراه کننده ارائه می‌دهند. از سوی دیگر، روش‌های صرفا داده‌محور ممکن است عملکرد بالایی داشته باشند اما فاقد تفسیرپذیری و سازگاری با رفتارهای شناخته‌شده حمل و نقلی هستند. در حال حاضر شکافی آشکار میان مدل‌هایی که هم سازگار با داده و هم مبتنی بر نظریه باشند، وجود دارد. این مطالعه درصدد پر کردن این شکاف است، از طریق معرفی شبکه عصبی آگاه از فیزیک که نمایش‌های درونی خود را با واقعیات فیزیکی نظیر بقای جریان، رشد یکنواحت در مناطق مرکزی و الگوهای توسعه شهری هماهنگ می‌سازد.
پیش‌بینی می‌شود که این پژوهش به تولید مدلی بینجامد که از لحاظ دقت، از روش‌های کلاسیک پیش‌بینی تقاضا عملکرد بهتری داشته باشد، به قوانین فیزیکی و رفتاری سیستم‌های حمل و نقل پایبند باشد، الگوهای قابل تفسیر در تحول تقاضای فضایی-زمانیی ارائه دهد و به عنوان لوک پایه‌ای در توسعه مدل‌های عصبی چهارمرحله‌ای آینده قابل استفاده باشد.
\subsection{فهرست نوآوری‌ها}
\begin{enumerate}
    \item ادغام اصول فیزیکی حمل و  نقل در معماری یادگیری عمیق
    \item استفاده از داده‌های نادر و طولی شهری برای پیش‌بینی زمانی
    \item توسعه چارچوب کلی شبکه عصبی آگاه بر فیزیک قابل اعمال در مراحل مختلف مدل چهارمرحله‌ای
    \item پر کردنشکاف میان مدل‌های پیش‌بینی جعبه سیاه و نظریه‌های قابل تفسیر حمل و نقل
    \item امکان بازتعریف زنجیره‌های مدل‌سازی تقاضای سفر با رویکردهای ترکیبی هوش مصنوعی
\end{enumerate}

% ---------------- Method ---------------------
\section{روش انجام پروژه}
در این بخش، روش‌ها و مراحل اجرایی تحقیق یا پروژه بیان می‌شود. جزئیات مربوط به نحوه جمع‌آوری داده‌ها، تحلیل‌ها و آزمایش‌ها ذکر می‌گردد.

% ---------------- Time Plan ---------------------
\section{برنامه زمانی}
در این بخش، برنامه زمانبندی انجام تحقیق یا پروژه، شامل مراحل مختلف آن و زمان‌بندی اجرای هر کدام آمده است.

% ---------------- Refrences ---------------------
\section{مراجع}
در این بخش، مراجع و منابع مورد استفاده در تحقیق یا پروژه آورده می‌شود.

\end{document}
